\documentclass[usenames,dvipsnames]{article}
\linespread{1.2}
\pagestyle{empty}
\usepackage{geometry}
 \geometry{
 a4paper,
 total={150mm,257mm},
 left=30mm,
 top=20mm,
 }
\usepackage[scaled]{uarial}
\usepackage{graphicx}
\graphicspath{ {./assets/} }
\usepackage{xcolor}
\definecolor{shadecolor}{RGB}{70,81,156}
\renewcommand*\familydefault{\sfdefault}
\usepackage[T1]{fontenc}
\usepackage{fontawesome}
\usepackage{ragged2e}
\usepackage{hyperref}
\hypersetup{
	colorlinks=true,
	linkcolor=true,
	urlcolor=NavyBlue
}



\title{First document}
\author{Gustav Lindblom}
\date{March 2020}

\begin{document}

\hspace{-1.6cm}\begin{minipage}[t]{0.65\textwidth}

\section*{\textcolor{NavyBlue}{Gustav Lindblom}}
\begin{flushleft}
En ambitiös backendutvecklare med 4 års erfarenhet av utveckling och förvaltning av större webbaserade platformar. Brinner för koncept som Clean Architecture och Clean Code när det kommer till mjukvarustruktur och design. Med flera års erfarenhet av applikationsutveckling har jag stark kompetens av .Net utveckling nom Microsofts diverse ekosystem.
\end{flushleft}

\section*{\textcolor{NavyBlue}{Anställningar}}
\subsection*{\textcolor{NavyBlue}{Pentia AB}}
\begin{flushleft}
\textsc{Head of Backend Development 2018 -- Pågående}\break
\textsc{Backend Consultant 2016 -- Pågående}\break 
\par Anställd i rollen som konsult inom backendutveckling för företagets kunder, med fokus på projekt byggda på Sitecore CMS. Gick 2018 även in i rollen som gruppledare för backendutveckling, med fokus på kompetensutveckling och stöd.
\par Ansvarig för nyutveckling, vidareutveckling, och förvaltning av existerande projekt, i lösningar utvecklade framförallt i Microsofts ASP.Net och MVC ramverk, grundat på Sitecore CMS.
\end{flushleft}

\subsection*{\textcolor{NavyBlue}{Adecco}}
\begin{flushleft}
\textsc{Konsult 2015 -- 2016}\break
Timanställd på konsultuppdrag för Øresundsbron som kundassistentmed fokus på snabbt och effektivt arbete ute i betalstationen tillØresundsbron. Direkt kundtjänst då det uppstår problem ibetalstationen, samt olika diversesysslor som krävs för att trafiken ska
flyta på. Treskiftsarbete, erfarenhet av långa kvällar och nätter
\end{flushleft}

\subsection*{\textcolor{NavyBlue}{Academic Work}}
\begin{flushleft}
\textsc{Konsult 2014 -- 2015}\break
Som ovan. Då Øresundsbron valde att använda sig av annatrekryteringsföretag för sitt behov av extra personal gavs jagmöjligheten att föra över min anställning till Adecco.
\end{flushleft}

\end{minipage}
\hspace{1.6cm}\begin{minipage}[t]{0.4\textwidth}
\vspace{0cm}
\begin{flushleft}
\begin{center}
\noindent\includegraphics[scale=0.25]{gustavlindblom}
\end{center}
\subsection*{\textcolor{NavyBlue}{Kontakt}}

\faPhone \qquad +46 735 06 80 61\\
\faEnvelope \qquad \href{mailto:lindblom.gustav@gmail.com}{lindblom.gustav@gmail.com}\\
\faLinkedin \qquad \href{https://www.linkedin.com/in/gustav-lindblom-a1529912a/}{LinkedIn}\\
\faGithub \qquad \href{https://github.com/glindblom}{GitHub}


\subsection*{\textcolor{NavyBlue}{Utbildning}}
\textsc{Systemutvecklare .Net}\\
Teknikhögskolan Malmö\\
2014 -- 2016

\subsection*{\textcolor{NavyBlue}{Certifieringar}}
\textsc{Sitecore Professional Developer Certification}\\
Utfärdat December 2016

\subsection*{\textcolor{NavyBlue}{Kompetenser}}
C\#\\
.Net\\
Sitecore\\
Typescript\\
Microsoft Azure Platform\\
DevOps CI/CD Pipelines \& Releases
\end{flushleft}
\end{minipage}


\end{document}