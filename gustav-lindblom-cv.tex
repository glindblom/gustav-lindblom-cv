\documentclass[usenames,dvipsnames]{article}
\linespread{1.2}
\pagestyle{empty}
\usepackage{geometry}
 \geometry{
 a4paper,
 total={150mm,257mm},
 left=30mm,
 top=20mm,
 }
\usepackage[scaled]{uarial}
\usepackage{graphicx}
\graphicspath{ {./assets/} }
\usepackage{xcolor}
\definecolor{shadecolor}{RGB}{70,81,156}
\renewcommand*\familydefault{\sfdefault}
\usepackage[T1]{fontenc}
\usepackage{fontawesome}
\usepackage{ragged2e}
\usepackage{hyperref}
\hypersetup{
	colorlinks=true,
	linkcolor=true,
	urlcolor=NavyBlue
}



\title{First document}
\author{Gustav Lindblom}
\date{March 2020}

\begin{document}

\hspace{-1.6cm}\begin{minipage}[t]{0.65\textwidth}

\section*{\textcolor{NavyBlue}{Gustav Lindblom}}
\begin{flushleft}
En ambitiös backendutvecklare med 4 års erfarenhet av utveckling och förvaltning av större webbaserade platformar utvecklade med ASP.Net, MVC, och Entity Framework, ovanpå Sitecores CMS platform. Brinner för koncept som Clean Architecture och Clean Code när det kommer till mjukvarustruktur och design. Med flera års erfarenhet av applikationsutveckling har jag stark kompetens av utveckling nom Microsofts diverse ekosystem, .Net, .Net Core, och Azure.\break
Programmeringsintresset började i förstahand med spelutveckling i tidig ålder, men har därefter utvecklats till att vara fokuserat framför allt på webben.
\end{flushleft}

\section*{\textcolor{NavyBlue}{Anställningar}}
\subsection*{\textcolor{NavyBlue}{Pentia AB}}
\begin{flushleft}
\textsc{Head of Backend Development 2018 -- Pågående}\break
\textsc{Backend Consultant 2016 -- Pågående}\break 
\par Anställd i rollen som konsult inom backendutveckling för företagets kunder, med fokus på projekt byggda på Sitecore CMS. Gick 2018 även in i rollen som gruppledare för backendutveckling, med fokus på kompetensutveckling och stöd.
\par Ansvarig för nyutveckling, vidareutveckling, och förvaltning av existerande projekt, i lösningar utvecklade framförallt i Microsofts ASP.Net och MVC ramverk, grundat på Sitecore CMS.
\end{flushleft}

\subsection*{\textcolor{NavyBlue}{Adecco}}
\begin{flushleft}
\textsc{Konsult 2015 -- 2016}\break
Timanställd på konsultuppdrag för Øresundsbron som kundassistentmed fokus på snabbt och effektivt arbete ute i betalstationen tillØresundsbron. Direkt kundtjänst då det uppstår problem ibetalstationen, samt olika diversesysslor som krävs för att trafiken ska
flyta på. Treskiftsarbete, erfarenhet av långa kvällar och nätter
\end{flushleft}

\subsection*{\textcolor{NavyBlue}{Academic Work}}
\begin{flushleft}
\textsc{Konsult 2014 -- 2015}\break
Som ovan. Då Øresundsbron valde att använda sig av annatrekryteringsföretag för sitt behov av extra personal gavs jagmöjligheten att föra över min anställning till Adecco.
\end{flushleft}

\end{minipage}
\hspace{1.6cm}\begin{minipage}[t]{0.4\textwidth}
\vspace{0cm}
\begin{flushleft}
\begin{center}
\noindent\includegraphics[scale=0.25]{gustavlindblom}
\end{center}
\subsection*{\textcolor{NavyBlue}{Kontakt}}

\faPhone \qquad +46 735 06 80 61\\
\faEnvelope \qquad \href{mailto:lindblom.gustav@gmail.com}{lindblom.gustav@gmail.com}\\
\faLinkedin \qquad \href{https://www.linkedin.com/in/gustav-lindblom-a1529912a/}{LinkedIn}\\
\faGithub \qquad \href{https://github.com/glindblom}{GitHub}


\subsection*{\textcolor{NavyBlue}{Utbildning}}
\textsc{Systemutvecklare .Net}\\
Teknikhögskolan Malmö\\
2014 -- 2016

\subsection*{\textcolor{NavyBlue}{Certifieringar}}
\textsc{Sitecore Professional Developer Certification}\\
Utfärdat December 2016

\subsection*{\textcolor{NavyBlue}{Kompetenser}}
C\#\\
.Net\\
Sitecore\\
Typescript\\
Microsoft Azure Platform\\
Agile \& Scrum Development\\
DevOps CI/CD Pipelines \& Releases
\end{flushleft}
\end{minipage}
\newpage
\section*{\textcolor{NavyBlue}{Uppgrad i fokus}}

\subsection*{\textcolor{NavyBlue}{Företag inom kemiindustri | 2018 --}}
\begin{flushleft}
Företaget hade ett större Sitecore-projekt där en ny design och struktur implementerades för hela plattformen. Jag arbetade som arkitekt och utvecklare på projektet och där hela plattformen behövde skrivas om från grunden. Dessutom anpassade jag plattformen till Helix vilket är best-practice inom Sitecore lösningar. \break
När den nya plattformen var färdig så har jag även arbetat med att uppdatera den till Sitecore 9.1 och anpassat plattformen för aktuell teknik vilket har gjort plattformen lättare att underhålla och använda för företaget. 

\end{flushleft}

\subsection*{\textcolor{NavyBlue}{Företag inom vatten \& avsloppshantering | 2017 -- 2019}}
\begin{flushleft}
Jag arbetade med två uppgraderingar av företagets Sitecore plattform och implementerade moderna lösningar för att gå ifrån föråldrad teknik. Till exempel skrev han om koden som hanterar plattformens sökfunktion och implementerade en Solr baserad lösning. 
\end{flushleft}

\subsection*{\textcolor{NavyBlue}{Företag inom mejeriproduktion | 2016 -- 2017}}
\begin{flushleft}
Företaget hade planerat en lansering av en ny hemsida och ett par månader innan release tog Gustav över utvecklingen och lanserade sidan. I projektet byggde han om hemsidan i Sitecore som tidigare varit baserat på ett annat CMS.  
\end{flushleft}

\subsection*{\textcolor{NavyBlue}{Företag inom reklamdistribution | 2016}}
\begin{flushleft}
Som del av ett utvecklingsteam arbetade Gustav med underhåll av en plattform med digitala reklamtjänster.
\end{flushleft}

\end{document}